\subsection{``需求牵引"}

云南省与缅甸、老挝、越南三国接壤,边境线长达4060公里,边境地区山川秀丽,被誉为“植物王国”和“世界花园”。
习近平总书记多次强调,``治国必治边",边境安防工作的重要性贯穿于国家发展的各个阶段。
然而,边境地区地形复杂多变,山地、高原、河谷等地貌交错分布且缺乏天然屏障,使得边境地区长期以来存在严重的非法越境隐患。
非法越境问题主要表现为人口非法流动、毒品走私、货物走私以及跨境犯罪等多种形式,该问题不仅直接影响边境地区的安全与稳定,还可能引发跨境犯罪、社会矛盾和民族问题,进而威胁国家整体安全和社会秩序。
近年来,云南边境地区非法越境问题愈发突出,给边境地区的社会稳定和国家的长治久安带来了巨大压力:

\begin{itemize}[left=15pt,itemsep=1em,label={\textasteriskcentered}]
\item \textbf{云南``1·04"组织外籍人员偷渡案。} 在2024年1月,云南与江苏警方联合侦破一起组织毗邻国家人员经陆路偷渡入境的案件。\textbf{犯罪团伙利用中缅、中越边境地形复杂多变的密林和隐蔽通道},组织外籍人员非法入境务工、通婚,同时衍生走私、涉毒等犯罪,查获走私物品案值310万元,缴获毒品798克,破坏边境社会稳定与经济安全\footnote{信息来源:江西公安公众号(\url{https://news.qq.com/rain/a/20250214A01KFX00})}(参见\reffig{fig:current-state}(a)-(b))。
\item \textbf{打洛边境派出所的防疫战。} 2017年起,打洛镇开始实施立体化管控,在陆地上修筑栅栏、建设抵边警务室,边境栅栏长达5公里,陆地上的偷渡渠道逐渐被封锁,越境路径全被挤压到了水路,因此河面设有雷达报警。在此基础上,民警定期轮岗和巡逻,\textbf{打造了立体式的监控网络},成为云南边防的典范。然而近年仍有非法越境的情况发生,例如,疫情期间因两例缅甸籍确诊病例非法入境,边境小城瑞丽不得不封城一周。从云南边境整体情况来看,2024年上半年云南边检机关共侦破妨害国(边)境管理刑事案件893起,抓获犯罪嫌疑人2930人,打掉组织偷渡团伙262个,\textbf{非法越境活动在云南边境地区仍然存在一定的规模}\footnote{信息来源:红星新闻(\url{https://news.qq.com/rain/a/20201005A08HYK00})}(参见\reffig{fig:current-state}(c)-(d))。
\item \textbf{云南``1·24"组织运送他人偷越国(边)境重特大案件。} 2020年至2021年期间,张某、匡某等11人不顾疫情防控政策,为获取高额报酬,违反出入国(边)境管理法规,\textbf{针对监控设施覆盖的漏洞和盲区,采取带路爬山、以摩托和轿车交互运输、用货物遮挡、绕道小路等方式逃避检查},组织、运送大批人员非法出、入境,更有涉案人员趁机实施运输毒品犯罪,严重破坏国(边)境管理和疫情防控工作\footnote{信息来源:云南长安网(\url{https://www.yncaw.gov.cn/html/2022/ftnw_0226/87066.html})}(参见\reffig{fig:current-state}(e)-(f))。
\end{itemize}

上述案件类型显示了执法部门在追捕犯罪嫌疑人和保护受害者方面的积极努力,但也反映出云南边境地区在打击非法越境问题方面仍存在提升空间。

\begin{figure}[h!]
\centering %图片居中
\includegraphics[width=1\textwidth]{1-1}
\captionsetup{justification=centering} %图题居中
\caption{云南边防现状}
\label{fig:current-state}
\end{figure}

习近平总书记在中共中央政治局第十八次集体学习时强调,``推进边疆治理体系和治理能力现代化,是中国式现代化的应有之义"。
打击非法越境是一个\textbf{超大场景下全天候的目标发现任务,监控系统需要在几十到几千米、甚至几十千米的较大地理空间范围内对目标进行监控,还需要实现7x24小时全气候条件下的监控覆盖}。
边境公安针对该问题正积极探索并应用前沿监控技术,以替代传统的警员巡逻模式,旨在克服传统手段中巡逻覆盖范围有限、反应速度滞后以及难以实现全天候、全方位监控等固有缺陷,同时有效应对日益隐蔽化和智能化的非法过境手段。然而,现有监控设备在技术层面仍存在显著不足,特别是缺乏高科技手段(如智能识别系统、多光谱成像、无人机协同巡查等)的深度集成,导致对非法过境行为的精准判断与打击能力未能达到预期效果。因此,目前超大场景安防监控仍有短板,并呈现如下现实需求:

\begin{itemize}[left=15pt,itemsep=1em,label={\textasteriskcentered}]
\item \textbf{需求1:适应超大场景复杂地形和动态环境变化的监控需求。} 从\textbf{云南``1·04"组织外籍人员偷渡案}可知,由于边境地区地形复杂多样,涵盖高山、丛林、河流等多种地貌,且环境动态变化频繁,导致监控设备在应对超大场景内物体随天气、时间变化导致的光线、能见度以及物体形态变化时,监控算法适应性和稳定性急剧下降或失效。这不仅限制了设备对环境的感知能力,还影响了监控系统的稳定性和可靠性,进而制约了边境监控的效能。因此,边境安防迫切需要开发能够适应复杂地形和动态环境变化的监控设备,以提升边境监控系统的整体性能和可靠性;
\item \textbf{需求2:超大场景下对潜在目标的实时预测和有效预判的监控需求。} 从\textbf{打洛边境派出所的防疫战}案例可知,现有监控设备在超大场景内的潜在目标主动预测与定位能力存在显著不足。当前监控系统主要依赖被动式监控模式,缺乏基于数据分析的主动预测机制,难以对非法越境目标的出现位置、行为轨迹及潜在风险进行实时预测与精准预判。这一技术短板导致边境安防工作长期处于被动监控模式,难以在非法活动发生前采取有效干预措施,从而降低了整体防控效能。因此,边境安防迫切需要研发具备主动预测与定位能力的监控系统,通过大数据分析和智能化技术,实现对潜在目标的实时预测和有效预判,从而提升边境安防的主动防御能力和整体防控效能;
\item \textbf{需求3:超大场景下主动搜索和智能调控的监控需求。} 从\textbf{云南``1·24"组织运送他人偷越国(边)境重特大案件}可知,现有监控设备体系存在明显的局限性。一方面,监控范围狭小导致在超大场景下的监控部署与运维成本居高不下,尤其是在云南省4060公里长的边境线上,单一摄像头的监控范围受限,加之山地、树林等物体的遮挡,难以实现无缝覆盖。另一方面,现有设备主要依赖被动式监控模式,无法通过调整摄像头的俯仰角、转动角及焦距主动搜索和发现超大场景下的目标,仍需依赖人力调整。因此,亟需研发具备宽广监控范围、主动搜索能力和智能化调控功能的监控系统,以降低监控部署与运维成本,进一步提升边境地区的整体防控效能。
\end{itemize}


综上,本项目旨在通过运用大数据分析与机器学习技术,\textbf{将人工干预的被动监控模式转变为算法实时控制的主动预测与搜索模式,提升监控设备在超大场景下的自主监控能力},努力满足上述三个方面的现实边防监控需求,从而有效治理非法越境问题,保障国家主权安全,推动边疆地区经济社会发展,为国家的稳定和发展创造良好环境。




















% 导致云南边境非法越境问题的原因是多方面的,涉及地理、经济、社会、政治等多个维度。首先,地理因素是非法越境问题的重要诱因。云南边境地形复杂,许多区域人迹罕至,难以实施全面监控,加之边境线漫长,有限的边防力量难以覆盖所有区域,这为非法越境活动提供了可乘之机。其次,经济发展不平衡是非法越境问题的重要驱动因素。云南边境地区与邻国如缅甸、老挝的经济发展水平存在较大差距,部分邻国居民为改善生活条件,选择非法进入中国。同时,边境地区部分居民生活贫困,容易受到非法活动的诱惑,参与走私、偷渡等行为。社会与文化因素也在其中扮演了重要角色。云南边境地区居住着多个跨境民族,如傣族、景颇族等,他们与邻国居民在语言、文化、血缘上有着密切联系,这种跨境联系为非法越境提供了便利。此外,部分边境地区社会治理能力薄弱,基层管理存在漏洞,难以有效遏制非法活动。

% 政治与法律因素同样不可忽视。邻国如缅甸长期存在政治动荡和武装冲突,导致大量难民和非法移民涌入云南边境,进一步加剧了非法越境问题的复杂性。跨境犯罪活动往往涉及多国,法律管辖和执法合作存在困难,难以形成有效的打击合力。技术与资源的限制也是边境管理面临的重要挑战。边境地区监控设施和技术手段相对落后,难以实现全天候、全方位的监控,而边防力量、资金和设备的分配也难以满足实际需求,导致部分边境区域管理薄弱。

% 云南边境非法越境问题不仅是一个区域性问题,更是涉及国家安全、社会稳定、经济发展和生态保护的综合性挑战。深入研究这一问题具有重要的理论和现实意义。从理论角度来看,通过多学科交叉研究,可以构建边境非法越境问题的理论框架,丰富边境治理和跨境管理的理论研究。从现实角度来看,研究成果可为政府制定边境管理政策、优化资源配置、提升治理能力提供科学依据,有助于维护边境安全、促进区域经济发展和社会稳定。因此,针对云南边境非法越境问题的研究具有重要的科学价值和社会意义。





